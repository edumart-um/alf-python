%!TEX root = Python.tex

\chapter{Documentación de Python}

\lettrine[lines=5]{F}{inalizamos} estos apuntes de Python con una explicación breve sobre el modo habitual de documentar el código. Si vas hacia atrás unas cuantas hojas, y más concretamente a la sección \ref{sec:comentarios} en la que tratábamos los comentarios de código en Python, podrás ver que el esqueleto que generó Eclipse para \texttt{helloword.py} incluía al comienzo una cadena de múltiples líneas que comenzaba y terminaba con tres comillas. Este tipo de cadena se conoce como \emph{docstring}. Hay ciertas partes del código de Python en las que podemos introducir estas cadenas para documentar lo que estamos haciendo. Por supuesto, a nadie le gusta documentar el código. ¡Que el que lo lea se esfuerce en entenderlo! Pero cuando te toca a ti darle vueltas a lo que ha escrito otro, no viene nada mal algunas palabritas sobre lo que ha hecho. 

En el código Python, al menos, deberíamos describir los módulos, las clases y las funciones. La documentación del módulo debe aparecer justo al comienzo del fichero, mientras que la de una clase o un método se indica justo a continuación de la línea \texttt{class} o la línea \texttt{def}. El esquema de la documentación es el siguiente:

\begin{lstlisting}
'''
Documentación del módulo
'''

class X:
    '''
    Documentación de la clase
    '''

    def metodo(args):
        '''
        Documentación de la función
        '''
\end{lstlisting}

En la siguiente sección indicamos sugerencias sobre lo que puede especificarse en cada cadena de documentación. No vamos a usar ningún sistema complejo para documentar el código, del estilo de \emph{reStructured}. Ten en cuenta que en esta asignatura no tienes que entregar una memoria de prácticas. A cambio, los profesores de la asignatura evaluamos que seas generoso en la documentación del código.

\section{Formato de la documentación}

La documentación de módulos, clases y funciones sigue un formato muy libre. Se realiza con cadenas multilínea (con tres comillas simples o dobles), y se recomienda que tenga las siguientes partes:
\begin{enumerate}
	\item Tras la apertura de la cadena, escribimos una línea en forma de resumen o título.
	\item Dejamos una línea en blanco después del resumen.
	\item A continuación escribimos una serie de líneas adicionales en las que nos podemos extender lo que queramos, evitando superar los 72 caracteres por línea.
	\item Después del cierre de la cadena, se deja una línea en blanco antes del código que viene a continuación.
\end{enumerate}

\subsection{Documentación de una clase}

Cuando describimos una clase, al menos deberíamos indicar lo siguiente:
\begin{itemize}
	\item Un breve resumen del propósito de la clase.
	\item Un listado de los atributos (de clase o instancia).
	\item Un listado de los métodos que queremos que se usen desde el código cliente.
\end{itemize}

Por ejemplo:

\begin{lstlisting}
class Afd:
    '''
    Clase que representa un fichero JFLAP en Python
	
    Atributos
    ---------
    nestados: int 
        Número de estados del autómata
    ntransiciones: int
        Número de transiciones del autómata

    Métodos
    -------
    getEstadoInicial() : str
        Devuelve el nombre del estado inicial
    esFinal(estado : str) : bool
        Devuelve True si el estado es final
    estadoSiguiente(estado : str, símbolo : str) : str
        Devuelve el destino de una transición
    '''
\end{lstlisting}

\subsection{Documentación de un método}

La documentación de los métodos puede ser más explícita a la hora de describir lo que estos hacen. Por ejemplo, se puede indicar:
\begin{itemize}
	\item Una breve descripción de lo que hace el método y cómo se usa.
	\item Una descripción de los argumentos, tanto requeridos como opcionales o nombrados.
	\item En el caso de los atributos opcionales, el valor por defecto que toman.
	\item Una descripción de cualquier excepción que se puede lanzar desde el método y en qué circunstancias.
\end{itemize}

Este podría ser un ejemplo:
\begin{lstlisting}
def estadoSiguiente(self, estado, símbolo)
    '''
    Devuelve un str con el destino de una transición desde un estado
    y con un símbolo dados. Si no existe una transición con esas 
    características, devuelve None.
	
    Parámetros
    ----------
    estado: str
        Estado desde el que se inicia la transición
    símbolo: str
        Símbolo que etiqueta la transición
\end{lstlisting}

\subsection{Documentación de un módulo}

En la parte superior de un módulo es conveniente escribir una descripción general del mismo que puede contener, entre otras cosas, lo siguiente:
\begin{itemize}
	\item Título del módulo y fecha de creación.
	\item Autor indicado tras \texttt{@author:}.
	\item Descripción general de las clases o métodos agrupados en el módulo y su funcionalidad común.
	\item Indicación de las dependencias del módulo, como los paquetes que deben estar instalados para poder usarlo.
	\item Indicación de la funcionalidad del módulo a través del código de primer nivel, en caso de que tenga un punto de entrada \texttt{\_\_main\_\_}.
\end{itemize}

\section{Uso de la documentación}

Al margen de permitir al que lee el código fuente una comprensión mayor sobre lo que realiza, podemos usar alguna herramienta que saca provecho de los \emph{docstrings} que hemos introducido en el código. Una de ellas es \texttt{help}. Desde el intérprete interactivo de Python podemos importar un módulo y consultar su documentación usando dicho comando. Por ejemplo:

\begin{lstlisting}
>>> import jflap.Afd
>>> help(jflap.Afd)
\end{lstlisting}
